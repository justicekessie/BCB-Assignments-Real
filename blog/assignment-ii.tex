% Options for packages loaded elsewhere
\PassOptionsToPackage{unicode}{hyperref}
\PassOptionsToPackage{hyphens}{url}
\PassOptionsToPackage{dvipsnames,svgnames,x11names}{xcolor}
%
\documentclass[
]{article}

\usepackage{amsmath,amssymb}
\usepackage{iftex}
\ifPDFTeX
  \usepackage[T1]{fontenc}
  \usepackage[utf8]{inputenc}
  \usepackage{textcomp} % provide euro and other symbols
\else % if luatex or xetex
  \usepackage{unicode-math}
  \defaultfontfeatures{Scale=MatchLowercase}
  \defaultfontfeatures[\rmfamily]{Ligatures=TeX,Scale=1}
\fi
\usepackage{lmodern}
\ifPDFTeX\else  
    % xetex/luatex font selection
\fi
% Use upquote if available, for straight quotes in verbatim environments
\IfFileExists{upquote.sty}{\usepackage{upquote}}{}
\IfFileExists{microtype.sty}{% use microtype if available
  \usepackage[]{microtype}
  \UseMicrotypeSet[protrusion]{basicmath} % disable protrusion for tt fonts
}{}
\makeatletter
\@ifundefined{KOMAClassName}{% if non-KOMA class
  \IfFileExists{parskip.sty}{%
    \usepackage{parskip}
  }{% else
    \setlength{\parindent}{0pt}
    \setlength{\parskip}{6pt plus 2pt minus 1pt}}
}{% if KOMA class
  \KOMAoptions{parskip=half}}
\makeatother
\usepackage{xcolor}
\setlength{\emergencystretch}{3em} % prevent overfull lines
\setcounter{secnumdepth}{-\maxdimen} % remove section numbering
% Make \paragraph and \subparagraph free-standing
\makeatletter
\ifx\paragraph\undefined\else
  \let\oldparagraph\paragraph
  \renewcommand{\paragraph}{
    \@ifstar
      \xxxParagraphStar
      \xxxParagraphNoStar
  }
  \newcommand{\xxxParagraphStar}[1]{\oldparagraph*{#1}\mbox{}}
  \newcommand{\xxxParagraphNoStar}[1]{\oldparagraph{#1}\mbox{}}
\fi
\ifx\subparagraph\undefined\else
  \let\oldsubparagraph\subparagraph
  \renewcommand{\subparagraph}{
    \@ifstar
      \xxxSubParagraphStar
      \xxxSubParagraphNoStar
  }
  \newcommand{\xxxSubParagraphStar}[1]{\oldsubparagraph*{#1}\mbox{}}
  \newcommand{\xxxSubParagraphNoStar}[1]{\oldsubparagraph{#1}\mbox{}}
\fi
\makeatother

\usepackage{color}
\usepackage{fancyvrb}
\newcommand{\VerbBar}{|}
\newcommand{\VERB}{\Verb[commandchars=\\\{\}]}
\DefineVerbatimEnvironment{Highlighting}{Verbatim}{commandchars=\\\{\}}
% Add ',fontsize=\small' for more characters per line
\usepackage{framed}
\definecolor{shadecolor}{RGB}{241,243,245}
\newenvironment{Shaded}{\begin{snugshade}}{\end{snugshade}}
\newcommand{\AlertTok}[1]{\textcolor[rgb]{0.68,0.00,0.00}{#1}}
\newcommand{\AnnotationTok}[1]{\textcolor[rgb]{0.37,0.37,0.37}{#1}}
\newcommand{\AttributeTok}[1]{\textcolor[rgb]{0.40,0.45,0.13}{#1}}
\newcommand{\BaseNTok}[1]{\textcolor[rgb]{0.68,0.00,0.00}{#1}}
\newcommand{\BuiltInTok}[1]{\textcolor[rgb]{0.00,0.23,0.31}{#1}}
\newcommand{\CharTok}[1]{\textcolor[rgb]{0.13,0.47,0.30}{#1}}
\newcommand{\CommentTok}[1]{\textcolor[rgb]{0.37,0.37,0.37}{#1}}
\newcommand{\CommentVarTok}[1]{\textcolor[rgb]{0.37,0.37,0.37}{\textit{#1}}}
\newcommand{\ConstantTok}[1]{\textcolor[rgb]{0.56,0.35,0.01}{#1}}
\newcommand{\ControlFlowTok}[1]{\textcolor[rgb]{0.00,0.23,0.31}{\textbf{#1}}}
\newcommand{\DataTypeTok}[1]{\textcolor[rgb]{0.68,0.00,0.00}{#1}}
\newcommand{\DecValTok}[1]{\textcolor[rgb]{0.68,0.00,0.00}{#1}}
\newcommand{\DocumentationTok}[1]{\textcolor[rgb]{0.37,0.37,0.37}{\textit{#1}}}
\newcommand{\ErrorTok}[1]{\textcolor[rgb]{0.68,0.00,0.00}{#1}}
\newcommand{\ExtensionTok}[1]{\textcolor[rgb]{0.00,0.23,0.31}{#1}}
\newcommand{\FloatTok}[1]{\textcolor[rgb]{0.68,0.00,0.00}{#1}}
\newcommand{\FunctionTok}[1]{\textcolor[rgb]{0.28,0.35,0.67}{#1}}
\newcommand{\ImportTok}[1]{\textcolor[rgb]{0.00,0.46,0.62}{#1}}
\newcommand{\InformationTok}[1]{\textcolor[rgb]{0.37,0.37,0.37}{#1}}
\newcommand{\KeywordTok}[1]{\textcolor[rgb]{0.00,0.23,0.31}{\textbf{#1}}}
\newcommand{\NormalTok}[1]{\textcolor[rgb]{0.00,0.23,0.31}{#1}}
\newcommand{\OperatorTok}[1]{\textcolor[rgb]{0.37,0.37,0.37}{#1}}
\newcommand{\OtherTok}[1]{\textcolor[rgb]{0.00,0.23,0.31}{#1}}
\newcommand{\PreprocessorTok}[1]{\textcolor[rgb]{0.68,0.00,0.00}{#1}}
\newcommand{\RegionMarkerTok}[1]{\textcolor[rgb]{0.00,0.23,0.31}{#1}}
\newcommand{\SpecialCharTok}[1]{\textcolor[rgb]{0.37,0.37,0.37}{#1}}
\newcommand{\SpecialStringTok}[1]{\textcolor[rgb]{0.13,0.47,0.30}{#1}}
\newcommand{\StringTok}[1]{\textcolor[rgb]{0.13,0.47,0.30}{#1}}
\newcommand{\VariableTok}[1]{\textcolor[rgb]{0.07,0.07,0.07}{#1}}
\newcommand{\VerbatimStringTok}[1]{\textcolor[rgb]{0.13,0.47,0.30}{#1}}
\newcommand{\WarningTok}[1]{\textcolor[rgb]{0.37,0.37,0.37}{\textit{#1}}}

\providecommand{\tightlist}{%
  \setlength{\itemsep}{0pt}\setlength{\parskip}{0pt}}\usepackage{longtable,booktabs,array}
\usepackage{calc} % for calculating minipage widths
% Correct order of tables after \paragraph or \subparagraph
\usepackage{etoolbox}
\makeatletter
\patchcmd\longtable{\par}{\if@noskipsec\mbox{}\fi\par}{}{}
\makeatother
% Allow footnotes in longtable head/foot
\IfFileExists{footnotehyper.sty}{\usepackage{footnotehyper}}{\usepackage{footnote}}
\makesavenoteenv{longtable}
\usepackage{graphicx}
\makeatletter
\def\maxwidth{\ifdim\Gin@nat@width>\linewidth\linewidth\else\Gin@nat@width\fi}
\def\maxheight{\ifdim\Gin@nat@height>\textheight\textheight\else\Gin@nat@height\fi}
\makeatother
% Scale images if necessary, so that they will not overflow the page
% margins by default, and it is still possible to overwrite the defaults
% using explicit options in \includegraphics[width, height, ...]{}
\setkeys{Gin}{width=\maxwidth,height=\maxheight,keepaspectratio}
% Set default figure placement to htbp
\makeatletter
\def\fps@figure{htbp}
\makeatother

<style>
  body {
    font-family: "Prima", sans-serif;
  }
</style>
\usepackage{mathpazo}
\usepackage[margin = 2.5cm]{geometry}
\makeatletter
\@ifpackageloaded{caption}{}{\usepackage{caption}}
\AtBeginDocument{%
\ifdefined\contentsname
  \renewcommand*\contentsname{Table of contents}
\else
  \newcommand\contentsname{Table of contents}
\fi
\ifdefined\listfigurename
  \renewcommand*\listfigurename{List of Figures}
\else
  \newcommand\listfigurename{List of Figures}
\fi
\ifdefined\listtablename
  \renewcommand*\listtablename{List of Tables}
\else
  \newcommand\listtablename{List of Tables}
\fi
\ifdefined\figurename
  \renewcommand*\figurename{Figure}
\else
  \newcommand\figurename{Figure}
\fi
\ifdefined\tablename
  \renewcommand*\tablename{Table}
\else
  \newcommand\tablename{Table}
\fi
}
\@ifpackageloaded{float}{}{\usepackage{float}}
\floatstyle{ruled}
\@ifundefined{c@chapter}{\newfloat{codelisting}{h}{lop}}{\newfloat{codelisting}{h}{lop}[chapter]}
\floatname{codelisting}{Listing}
\newcommand*\listoflistings{\listof{codelisting}{List of Listings}}
\makeatother
\makeatletter
\makeatother
\makeatletter
\@ifpackageloaded{caption}{}{\usepackage{caption}}
\@ifpackageloaded{subcaption}{}{\usepackage{subcaption}}
\makeatother

\ifLuaTeX
  \usepackage{selnolig}  % disable illegal ligatures
\fi
\usepackage{bookmark}

\IfFileExists{xurl.sty}{\usepackage{xurl}}{} % add URL line breaks if available
\urlstyle{same} % disable monospaced font for URLs
\hypersetup{
  pdftitle={Assignment II},
  colorlinks=true,
  linkcolor={blue},
  filecolor={Maroon},
  citecolor={Blue},
  urlcolor={Blue},
  pdfcreator={LaTeX via pandoc}}


\title{Assignment II}
\author{}
\date{}

\begin{document}
\maketitle


\subsection{Description of Data}\label{description-of-data}

The covid\_ts dataset tracks the daily counts of COVID-19 infections and
deaths across U.S. states, broken down by different virus strains. It
includes several key columns: the date, state, total infections, deaths,
and daily case counts for each COVID-19 strain--- \emph{Original},
\emph{Alpha}, \emph{Delta}, and \emph{Omicron}. The date column records
the day of the entry, while the state column tells you which U.S. state
the data pertains to. The total\_infections column gives the overall
count of COVID-19 cases, and deaths shows the number of lives lost to
the virus. The strain-specific columns break down the daily cases for
each variant: \emph{Original}, \emph{Alpha}, \emph{Delta}, and
\emph{Omicron}.

The dataset begins in early 2020, during the first wave of the pandemic,
and continues through the various waves caused by the emergence of new
strains. These variants, like Alpha, Delta, and Omicron, had different
levels of contagiousness and impacts on public health, which is
reflected in their infection and death rates. For instance, the Alpha
variant, emerging in late 2020, was more easily spread than the original
strain, while Delta, which appeared in 2021, caused a major global
surge. Omicron, emerging toward the end of 2021, was highly
transmissible but led to fewer severe cases compared to previous
strains.

By analyzing this dataset, we can track how each variant's dominance
shifted over time and across regions. It provides insight into when and
where each strain became the most prevalent, shedding light on regional
variations in how COVID-19 spread. Additionally, the dataset allows us
to explore how mortality rates changed with each variant. By comparing
death counts for each strain, we can gain a better understanding of how
the severity of the virus evolved. Overall, the dataset is a powerful
tool for both public health policy makers and epidemiological
researchers. It allows us to visualize how the virus spread
geographically and how different variants impacted different regions at
various times. For example, using pie charts, we can easily show the
proportion of each strain in different states, offering a clear picture
of how the pandemic unfolded across the U.S. over time.

\begin{Shaded}
\begin{Highlighting}[]
\NormalTok{\# Required packages}
\NormalTok{library(tidyverse)}
\NormalTok{library(lubridate)}
\NormalTok{library(maps)}
\NormalTok{library(gganimate)}
\NormalTok{library(gifski)}
\end{Highlighting}
\end{Shaded}

```\{r, include=FALSE\} \#\textbar{} echo: false \#set.seed(123)

\section{Simplified state coordinates data with proper region
conversion}\label{simplified-state-coordinates-data-with-proper-region-conversion}

states \textless- tibble( state = state.name, longitude =
state.center\(x,
  latitude = state.center\)y, region = as.character(state.region) )

\section{Time parameters}\label{time-parameters}

start\_date \textless- as.Date(``2020-01-01'') end\_date \textless-
as.Date(``2022-12-31'') dates \textless- seq(start\_date, end\_date, by
= ``day'')

\section{Strain configuration}\label{strain-configuration}

strains \textless- c(``Original'', ``Alpha'', ``Delta'', ``Omicron'')
strain\_emergence \textless- list( Original = as.Date(``2020-01-01''),
Alpha = as.Date(``2020-09-01''), Delta = as.Date(``2021-03-01''),
Omicron = as.Date(``2021-11-01'') )

calculate\_strain\_probs \textless- function(date, region) \{ date
\textless- as.Date(date) region \textless- as.character(region)

region\_modifier \textless- case\_when( region == ``Northeast''
\textasciitilde{} 1.2, region == ``South'' \textasciitilde{} 1.0, region
== ``North Central'' \textasciitilde{} 0.9, region == ``West''
\textasciitilde{} 1.1, TRUE \textasciitilde{} 1.0 )

get\_strain\_prob \textless- function(strain\_date, fitness) \{
strain\_date \textless- as.Date(strain\_date) days\_since\_emergence
\textless- as.numeric(difftime(date, strain\_date, units = ``days'')) if
(days\_since\_emergence \textless{} 0) return(0) 1 / (1 + exp(-fitness *
region\_modifier * (days\_since\_emergence/100 - 3))) \}

probs \textless- c( Original =
get\_strain\_prob(strain\_emergence\(Original, 0.8),
    Alpha = get_strain_prob(strain_emergence\)Alpha, 1.0), Delta =
get\_strain\_prob(strain\_emergence\(Delta, 1.2),
    Omicron = get_strain_prob(strain_emergence\)Omicron, 1.5) )

probs \textless- pmax(0, probs + rnorm(length(probs), 0, 0.02)) if
(sum(probs) == 0) probs \textless- c(1, 0, 0, 0) probs / sum(probs) \}

\begin{verbatim}

```{r}
# Generate data with proper strain distribution
covid_ts <- tibble()

for(s in states$state) {
  state_region <- as.character(states$region[states$state == s])
  
  for(d in dates) {
    d <- as.Date(d)
    
    day_of_year <- yday(d)
    seasonal_factor <- 1 + 0.3 * sin(2 * pi * (day_of_year - 30)/365)
    base_rate <- 100 * seasonal_factor
    
    # Calculate probabilities for each strain
    probs <- calculate_strain_probs(d, state_region)
    
    # Generate total infections
    total_infections <- rpois(1, lambda = base_rate)
    
    # Distribute infections across strains
    if(total_infections > 0) {
      strain_counts <- rmultinom(1, total_infections, probs)[,1]
    } else {
      strain_counts <- rep(0, length(strains))
    }
    names(strain_counts) <- strains
    
    # Calculate deaths for each strain separately
    mortality_rates <- c(Original = 0.02, Alpha = 0.025, Delta = 0.035, Omicron = 0.01)
    deaths <- sum(sapply(1:length(strains), function(i) {
      rbinom(1, strain_counts[i], mortality_rates[strains[i]])
    }))
    
    # Create daily record with proper strain distribution
    daily_data <- tibble(
      date = d,
      state = s,
      total_infections = total_infections,
      deaths = deaths,
      Original = strain_counts["Original"],
      Alpha = strain_counts["Alpha"],
      Delta = strain_counts["Delta"],
      Omicron = strain_counts["Omicron"]
    )
    
    covid_ts <- bind_rows(covid_ts, daily_data)
  }
  
  cat("Processed state:", s, "\n")
}

# Verify that strains sum to total infections
check_sums <- covid_ts %>%
  mutate(
    strain_sum = Original + Alpha + Delta + Omicron,
    matches = strain_sum == total_infections
  )
\end{verbatim}

\begin{Shaded}
\begin{Highlighting}[]
\NormalTok{\# Print summary to verify strain distribution}
\NormalTok{summary\_stats \textless{}{-} covid\_ts \%\textgreater{}\%}
\NormalTok{  group\_by(state) \%\textgreater{}\%}
\NormalTok{  summarise(}
\NormalTok{    total\_cases = sum(total\_infections, na.rm = TRUE),}
\NormalTok{    total\_original = sum(Original, na.rm = TRUE),}
\NormalTok{    total\_alpha = sum(Alpha, na.rm = TRUE),}
\NormalTok{    total\_delta = sum(Delta, na.rm = TRUE),}
\NormalTok{    total\_omicron = sum(Omicron, na.rm = TRUE)}
\NormalTok{  )}
\NormalTok{print(summary\_stats)}
\end{Highlighting}
\end{Shaded}

\texttt{\{r,\ include=FALSE\}\ \#\ Create\ visualization\ of\ dominant\ strains\ \#\ Modify\ the\ dominant\ strains\ data\ to\ ensure\ unique\ state-month-year\ combinations\ dominant\_strains\ \textless{}-\ covid\_ts\ \%\textgreater{}\%\ \ \ mutate(\ \ \ \ \ year\ =\ year(date),\ \ \ \ \ month\ =\ month(date)\ \ \ )\ \%\textgreater{}\%\ \ \ group\_by(state,\ year,\ month)\ \%\textgreater{}\%\ \ \ summarize(\ \ \ \ \ Original\ =\ sum(Original),\ \ \ \ \ Alpha\ =\ sum(Alpha),\ \ \ \ \ Delta\ =\ sum(Delta),\ \ \ \ \ Omicron\ =\ sum(Omicron),\ \ \ \ \ .groups\ =\ \textquotesingle{}drop\textquotesingle{}\ \ \ )\ \%\textgreater{}\%\ \ \ pivot\_longer(\ \ \ \ \ cols\ =\ all\_of(strains),\ \ \ \ \ names\_to\ =\ "strain",\ \ \ \ \ values\_to\ =\ "count"\ \ \ )\ \%\textgreater{}\%\ \ \ group\_by(state,\ year,\ month)\ \%\textgreater{}\%\ \ \ slice\_max(count,\ n\ =\ 1,\ with\_ties\ =\ FALSE)\ \%\textgreater{}\%\ \ \ ungroup()}

```\{r, include = FALSE\} \# Clean up state names in map data us\_map
\textless- map\_data(``state'') \%\textgreater\% mutate( state =
str\_to\_title(region), \# Create a unique identifier for each polygon
piece piece\_id = group )

\section{Join the data with explicit relationship
specification}\label{join-the-data-with-explicit-relationship-specification}

map\_data \textless- us\_map \%\textgreater\% left\_join(
dominant\_strains, by = ``state'', relationship = ``many-to-many'' )

\begin{verbatim}

```{r}
# Create the visualization with updated aesthetics
p <- ggplot(map_data, aes(x = long, y = lat, group = piece_id)) +
  geom_polygon(
    aes(fill = strain),
    color = "white",
    linewidth = 0.2  # Updated from 'size' to 'linewidth'
  ) +
  coord_map("albers", lat0 = 39, lat1 = 45) +
  scale_fill_brewer(palette = "Set3") +
  labs(
    title = "COVID-19 Strain Distribution in the United States",
    subtitle = "Date: {frame_time}",
    fill = "Dominant Strain"
  ) +
  theme_minimal() +
  theme(
    legend.position = "bottom",
    plot.title = element_text(size = 14, face = "bold"),
    plot.subtitle = element_text(size = 12),
    panel.grid = element_blank()  # Remove grid lines for cleaner map
  ) +
  transition_time(as.Date(paste(year, month, "01", sep = "-"))) +
  ease_aes('linear')
\end{verbatim}

\begin{Shaded}
\begin{Highlighting}[]
\NormalTok{p}
\end{Highlighting}
\end{Shaded}

\begin{Shaded}
\begin{Highlighting}[]

\end{Highlighting}
\end{Shaded}





\end{document}
